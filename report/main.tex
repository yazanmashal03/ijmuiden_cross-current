\documentclass[12pt,a4paper]{article}
\usepackage[utf8]{inputenc}
\usepackage[T1]{fontenc}
\usepackage{amsmath,amsfonts,amssymb}
\usepackage{graphicx}
\usepackage{float}
\usepackage{hyperref}
\usepackage{geometry}
\usepackage{booktabs}
\usepackage{listings}
\usepackage{xcolor}
\usepackage{cite}

\geometry{margin=1in}

\title{Cross-Current Forecasting for the Port of Amsterdam: \\
A Machine Learning Approach Using LSTM Networks}
\author{Technical Report}
\date{\today}

\begin{document}

\maketitle

\begin{abstract}
This report presents a comprehensive machine learning solution for cross-current forecasting at the Port of Amsterdam. The project addresses the critical need for accurate maritime navigation predictions by developing an LSTM-based neural network model that integrates multiple environmental variables including wind direction, wind speed, water height, wave height, and cross-current measurements. Our approach achieves significant improvements in prediction accuracy through advanced feature engineering, robust data preprocessing, and optimized neural network architecture. The model demonstrates strong performance with mean squared error reduction and provides reliable forecasts for maritime operations planning.
\end{abstract}

\tableofcontents
\newpage

\section{Introduction}

\subsection{Background and Motivation}

\subsection{Problem Statement}

\subsection{Project Objectives}

\section{Data Description and Preprocessing}

\subsection{Data Sources}

\subsection{Data Characteristics}

\subsection{Data Preprocessing Pipeline}

\subsubsection{Data Ingestion}

\subsubsection{Data Cleaning}

\subsubsection{Normalization}

\subsubsection{Feature Engineering}

\section{Model Architecture}

\subsection{LSTM Network Design}

\subsubsection{Architecture Overview}

\subsubsection{Mathematical Formulation}

\subsubsection{Activation Functions}

\subsection{Training Configuration}

\section{Training and Validation}

\subsection{Data Splitting Strategy}

\subsection{Training Process}

\subsubsection{Model Initialization}

\subsubsection{Regularization Techniques}

\subsubsection{Learning Rate Scheduling}

\subsection{Model Selection}

\section{Results and Evaluation}

\subsection{Performance Metrics}

\subsection{Feature Importance Analysis}

\subsection{Model Interpretability}

\section{Inference System}

\subsection{Production Pipeline}

\subsubsection{Data Preprocessing}

\subsubsection{Model Inference}

\subsection{Performance Optimization}

\section{Future Research Directions}

\subsection{Model Enhancements}


\subsubsection{Architecture Improvements}

\subsubsection{Data Integration}

\subsection{Advanced Techniques}

\subsection{Operational Improvements}

\section{Conclusion}

\subsection{Impact and Applications}

\end{document}